\documentclass{ama}

% Load custom packages here, e.g.:
\usepackage{pgfplots}
\pgfplotsset{compat=1.18}

\addbibresource{refs.bib}

% Modify the topic identifier according to the notification of acceptance.
% The topic identifier is for reference only and will be removed in the final publication.
\topic{X-42}

\title{Proceedings Preparation Guidelines for the 2025 International Congress on Ultrasonics}

\author[1]{First Author}
\author[1]{\underline{Presenting Author}}
\author[2]{Third Author}

\affil[1]{Institution or company, address and country}
\affil[2]{Institution or company, address and country}

\email{corresponding.authors@e-mail.example}

\abstract{%
    Briefly describe the highlights of your work in a maximum of 6 lines.
    Please do not use symbols and references to literature in the summary.
}

\keywords{%
    List five descriptive keywords, separated by a comma.
    These will be included in the index of keywords in the electronic conference volume.
}

\begin{document}

\section{Template}
The template for the proceedings is available \href{https://github.com/emtpb/icu_proceedings/archive/refs/heads/main.zip}{\underline{here}}.
Should you encounter any issues, please report them at the templates source code repository at \href{https://github.com/emtpb/icu_proceedings}{\underline{github.com/emtpb/icu\_proceedings}}.

\section{Title and Headlines}
In the main title, please use initial capital letters.
Do not capitalize articles (like `the'), coordinate conjunctions (`the'), and prepositions (`of', `in') under four letters in length.
Underline the presenting author.
Please only use sections to structure the manuscript.

\section{Background, Motivation and Objective}
To introduce your topic please give a short background motivation and refer to selected relevant literature.
Take care that your contribution is in line with the suggested topic on top of the page.
Choose an adequate title that reveals the aim of your work.
Experimental details should be shown only as distinctive features to describe important information.

\section{Results}
Only original scientific results and significant findings and developments in the specific field should be presented.
Please arrange figures so that important information can be seen at a glance.
Sometimes a picture is worth a thousand words.

\section{Compiling}
The document can be compiled using the typical routine of \LaTeX{} commands:
%
\begin{verbatim}
    pdflatex main.tex
    biber main
    pdflatex main.tex
    pdflatex main.tex
\end{verbatim}
%
Make sure to use \texttt{biber} to build the bibliography.
The \texttt{pdflatex} and \texttt{biber} commands should be available on all common \LaTeX{} distributions.
To perform all the necessary steps in a single command, \texttt{latexmk} can be used if available:
%
\begin{verbatim}
    latexmk -pdf main.tex
\end{verbatim}
%
Please ensure that all references to literature and figures are displayed correctly before submitting the document.

\section{Pagination}
Please do not put page numbers on the paper.
%
\begin{figure}
    \centering
    \begin{tikzpicture}[font=\footnotesize]

\begin{axis}[
width=\linewidth,
xlabel={Concentration \(c_\mathrm{N\textsubscript{2}O}\) / \si{ppm}},
xmin=0, xmax=10,
ylabel={Quantity \(q\) / \si{Unit}},
ymin=0, ymax=10,
legend pos=south east
]
\addplot[only marks, mark=*]
table{%
x y
0 0
1 1.07
2 2.00
3 2.88
4 3.99
5 4.99
6 6.21
7 7.01
8 7.90
9 9.04
10 10
};\addlegendentry{\SI{400}{\celsius}}

\addplot[only marks, mark=square*, color=gray]
table{%
x y
0 0
1 0.77
2 1.62
3 2.38
4 3.27
5 4.00
6 4.81
7 5.60
8 6.51
9 7.20
10 8
};\addlegendentry{\SI{300}{\celsius}}

\addplot[thick]
table{%
x y
0 0
10 10
};

\addplot[thick, color=gray, dashed]
table{%
x y
0 0
10 8
};

\node[align=left, anchor=north west] at (axis cs:0.25,9.75) {%
    Base gas: \SI{20}{\percent} O\textsubscript{2} in N\textsubscript{2}
};
\node[align=left, anchor=north west] at (axis cs:0.25,8.5) {%
    Self-heated sensor element\\
    Pt-doped functional material
};
\end{axis}

\end{tikzpicture}

    \caption{Graph of variable vs analyte concentration for two different temperatures.}\label{fig:ex1}
\end{figure}

\section{Equations}
Equations should appear in a separate line.
Please refer your equations in the text as \autoref{eq:ex1}.
%
\begin{equation}
    R = \frac{U}{I}\label{eq:ex1}
\end{equation}

\section{Graphics, Photographs, and Tables}
Please arrange figures in a way that important information can be caught at a glance.
Illustrations, graphs, and photographs should be centred, preferably fit in one column (see \autoref{fig:ex1}), and either be placed at the top \texttt{[t]} or at the bottom \texttt{[b]} of a column.
The resolution for photographs should not exceed 300 dpi.
The use of colour is encouraged, but please make sure that the figures can still be interpreted when converted to greyscale.

Captions are placed above tables (i.e.\ \autoref{tab:ex1}) and below figures (i.e.\ \autoref{fig:ex1}), picture, or graphic.
%
\begin{table}
    \caption{Table captions should be place above the table they are describing.}\label{tab:ex1}
    \centering
    \begin{tabular}{ccc}
        \toprule
        \(T / \si{\celsius}\) & Sensitivity of & Coefficient of \\
        & variable towards & determination \(R^2\)  \\
        & N\textsubscript{2}O & \\
        \midrule
        \num{300} & \num{0.8} & \num{0.9998} \\
        \midrule
        \num{400} & \num{1} & \num{0.9994} \\
        \bottomrule
    \end{tabular}
\end{table}

\section{References}
List and number all bibliographical references at the end of the paper.
When referenced within the text, enclose the citation number in square brackets, i.e.~\cite{mouse2011}.
The correct style will be used automatically if you choose to use Bib\LaTeX{} to generate the bibliography.

\printbibliography{}

\section{Please Note}
Your document is expected to have a total length of four (4) pages.
Company logos may not be included in the short paper.
By submitting your short paper you allow the organizer to publish your submission in the electronic conference proceedings and on the web.
The papers will receive a Digital Object Identifier (DOI).
The Authors are responsible for obtaining any permissions which may be required from governments, industries and/or other organizations.
No responsibility is accepted by the organizer.
Work that is not original or that has been presented in other international conferences or publications will not be accepted unless considerable progress has been achieved.

\end{document}
